\chapter{Introduction}
\section{Background of the Study}
 Polycystic ovarian syndrome (PCOS) is a common endocrine, metabolic, and reproductive disorder in women of reproductive age, stemming from an interplay of multiple factors (Diamanti-Kandarakis \& Papavassiliou, 2006; Zhou et al., 2021). Though the precise pathogenesis of PCOS remains obscure, genetics, epigenetics, and the environment have been pinpointed as central to PCOS manifestation. 
 
 At present, an estimate of 6-21\% of women worldwide are afflicted by the syndrome and exhibit a wide range of symptoms at varying degrees (Dunaif, 1997; Zhao et al., 2023). PCOS is characterized by both reproductive and metabolic abnormalities including polycystic ovary morphology, dysfunctional ovulation, and hyperandrogenism alongside insulin resistance (IR) and obesity. Given the heterogeneity of PCOS, four phenotypes have been defined: type A, polycystic ovary, chronic oligo-anovulation and hyperandrogenism; type B, chronic oligo anovulation and hyperandrogenism; type C, polycystic ovary and hyperandrogenism; and type D, polycystic ovary, chronic oligo-anovulation (Dunaif, 1997; Zhao et al., 2023). Additionally, IR—though varying in degree–is present in all phenotypes as 65-95\% of PCOS-affected women regardless of weight suffer from it and compensatory hyperinsulinemia (Zhao et al., 2023). 

Insulin resistance is essentially defined by a decrease in insulin sensitivity. In effect, IR impairs insulin action in its target tissues (adipose, skeletal muscle, liver, ovarian, and uterine tissue), subsequently resulting in compensatory hyperinsulinemia. Specifically, IR hampers insulin-stimulated glucose transport and inhibits lipolysis in adipose tissue; decreases in glucose transport and muscle glycogen synthesis in skeletal muscle; impairs gluconeogenesis suppression but stimulates fatty acid synthesis in liver tissue; and, induces androgen-dependent anovulation in the ovaries and uterus (Hardy et al., 2012; Zhao et al., 2023). In the insulin signal transduction pathway of PCOS women, it has been hypothesized that increased serine phosphorylation and decreased tyrosine phosphorylation of insulin receptors and insulin receptor substrate proteins (IRS) can terminate insulin action (Diamanti-Kandarakis \& Dunaif, 2012). Thus, despite the unaltered affinity of insulin to its receptor, insulin dysfunction may still result due to post-binding defects in insulin signal transduction.

Protein tyrosine phosphatase 1B (PTP1B) normally catalyzes the dephosphorylation of the insulin receptor, thus decreasing insulin-stimulated receptor tyrosine kinase activity. Consequently, an increase of PTP1B activity has been correlated with a loss of insulin signaling (Chen et al., 2010). Similarly, aberrant insulin signaling has also been associated with decreased receptor-mediated insulin substrate tyrosine phosphorylation and an increase of serine phosphorylation due to the action of the inhibitor kappa B kinase (IKK-$\beta$) complex––the second protein target of the present study. Due to the IKK-mediated serine phosphorylation of IRS-1, the protein cannot activate PI3K, thereby halting the subsequent steps in the insulin signaling pathway (Gao et al., 2002).

If left untreated, women with PCOS are more likely to experience adverse pregnancy outcomes such as spontaneous abortions, preterm birth, gestational diabetes mellitus, and pregnancy-induced hypertension (Pattnaik et al., 2022). Furthermore, there is an increased risk of developing type 2 diabetes mellitus, cardiovascular disease, and uterine cancer (Polycystic Ovary Syndrome (PCOS), 2022). 

Oral contraceptives are commonly used to aid in regularizing menstrual periods and mitigating hyperandrogenism, though some studies have found that these agents––especially those containing triphasic progestin––may produce or worsen IR (Dunaif, 1997). Metformin, on the other hand, suppresses hepatic glucose production and improves insulin sensitivity by mediating weight loss. Nevertheless, it does not help manage androgen levels nor does it alter insulin action itself (Dunaif, 1997; Huang et al., 2016; Zhao et al.,  2023). Thiazolidinediones (TZDs) like troglitazone and pioglitazone are true insulin sensitizers that target peroxisome proliferator activated receptor gamma (PPAR-$\gamma$), which mediates insulin activity. Unlike metformin, however, TZDs do so without altering body weight while also lowering androgen, estrogen, and luteinizing hormone levels in PCOS (Huang et al., 2016; Zhao et al., 2023). Similarly, myo-inositol, a sugar-alcohol supplement, has been observed to exhibit insulin sensitization efficiency while also promoting ovulation (Zhao et al.,  2023). 

Lastly, berberine, a benzylisoquinoline alkaloid derivative isolated from Traditional Chinese medicinal herbs, is known to alleviate metabolic disorders. Berberine itself participates in several insulin signaling pathways (i.e., PPAR, MAPK, AMPK), making it a novel PCOS treatment (Zhao et al., 2023). In relation to the aforementioned protein targets, BRR has the potential to bind PTP1B, inhibiting its catalytic activity which consequently enhances the phosphorylation of insulin receptor and insulin receptor substrate-1 (IRS-1) in 3T2-L1 adipocytes (Zhang et al., 2021). Furthermore, a previous study involving gestational diabetes mellitus (GDM) rats found that the binding of BRR to IKK-$\beta$ results in decreased IR due to the modulation of different signaling pathways in the liver (Li et al., 2022). In effect, by activating insulin signaling pathways and promoting the utilization of glucose, berberine and perhaps its functional derivatives may have the potential to manage IR that exacerbates PCOS progression.

\section{Statement of the Problem}
What benzylisoquinoline alkaloid derivatives have the greatest binding affinities to PTP1B and IKK-$\beta$ and possess suitable chemical properties for drug development, thus becoming effective inhibitors towards insulin resistance?  

\section{Research Objectives}
The aim of the present study is to identify and design benzylisoquinoline alkaloid-based drug leads targeting protein tyrosine phosphatase 1B (PTP1B) and inhibitor kappaB kinase (IKK) complexes as treatments for PCOS insulin resistance. 

Specifically, the researchers aim to: 
\begin{itemize}
     \item Identify benzylisoquinoline alkaloid derivatives that effectively bind to PTP1B and IKK-$\beta$ via structure-based virtual screening.
     \item Determine the binding affinity of top-hit benzylisoquinoline alkaloid derivatives to PTP1B and IKK-$\beta$ respectively through molecular docking. 
    \item Optimize benzylisoquinoline alkaloid-based ligand using QSAR results.
    \item Describe protein-ligand interactions between initial top binding compounds and modified benzylisoquinoline alkaloid ligands with the PTP1B and IKK-$\beta$ through computer-aided visualization. 
    \item Evaluate the top candidate ligands based on physicochemical and pharmacokinetic properties using ADMETox assessments.
    \item Determine if best-binding modified benzylisoquinoline alkaloids may be naturally sourced (i.e., from plants, microorganisms, animals, etc.).
\end{itemize}

\section{Significance of the Study}
This study will provide essential baseline information on the viability of berberine derivatives as a form of treatment for PCOS insulin resistance by binding specifically to PTP1B and IKK. Moreover, the availability of information about berberine’s effect on PCOS pathways remains scarce as it has only gained interest in the field recently. Data generated throughout the study will be able to supplement existing evidence that the intake of berberine decreases insulin resistance and improves the overall menstrual patterns in infertile PCOS women. 

\section{Scope and Limitations}
The study aims to identify the most promising benzylisoquinoline alkaloid derivative with high potential for therapeutic effects against IR manifested in PCOS women when bound separately to PTP1B and IKK target proteins. Computational analysis tools, such as MTiOpenScreen web server, AutoDock Tools, GROMACS, MarvinSketch,3DQSAR, LigPlot+, and ADMETox will be employed to design benzylisoquinoline alkaloid-based ligands and determine the key interactions that occur in the generated models. \textit{In vitro} and \textit{in vivo} experiments are beyond the scope of this study.