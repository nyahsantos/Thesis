Ajmal, N., Khan, S., & Shaikh, R. (2019). Polycystic ovary syndrome (PCOS) and genetic predisposition: A review article. European Journal of Obstetrics & Gynecology and Reproductive Biology: X, 3, 100060–100060. https://doi.org/10.1016/j.eurox.2019.100060

Bateman, A., Martin, M., Orchard, S., Magrane, M., Ahmad, S., Alpi, E., Bowler-Barnett, E., Britto, R., Bye-A-Jee, H., Cukura, A., Denny, P., Doğan, T., Ebenezer, T. E., Fan, J., Garmiri, P., Gonzales, L., Hatton-Ellis, E., Hussein, A., Ignatchenko, A., . . . Zhang, J. (2022). UniProt: the Universal Protein Knowledgebase in 2023. Nucleic Acids Research, 51(D1), D523–D531. https://doi.org/10.1093/nar/gkac1052

Chang, X., Yan, H., Xu, Q., Xia, M., Bian, H., Zhu, T., & Gao, X. (2012). The effects of berberine on hyperhomocysteinemia and hyperlipidemia in rats fed with a long-term high-fat diet. Lipids in Health and Disease, 11(1). https://doi.org/10.1186/1476-511x-11-86

Chen, C., Zhang, Y., & Huang, C. (2010). Berberine inhibits PTP1B activity and mimics insulin action. Biochemical and Biophysical Research Communications, 397(3), 543–547. https://doi.org/10.1016/j.bbrc.2010.05.153

Christakou, C. D., & Diamanti-Kandarakis, E. (2008). Role of androgen excess on metabolic aberrations and cardiovascular risk in women with polycystic ovary syndrome. Women’s Health, 4(6), 583-594.

Delibegović, M., & Mody, N. (2013). PTP1B in the periphery: Regulating insulin sensitivity and ER stress. In Springer eBooks. https://doi.org/10.1007/978-1-4614-7855-3\_5

Diamanti-Kandarakis, E., & Dunaif, A. (2012). Insulin Resistance and the Polycystic Ovary Syndrome Revisited: An update on Mechanisms and Implications. Endocrine Reviews, 33(6), 981–1030. https://doi.org/10.1210/er.2011-1034

Diamanti-Kandarakis, E., & Papavassiliou, A. G. (2006). Molecular mechanisms of insulin resistance in polycystic ovary syndrome. Trends in Molecular Medicine, 12(7), 324–332. https://doi.org/10.1016/j.molmed.2006.05.006

Dumesic, D. A., Padmanabhan, V., Chazenbalk, G. D., & Abbott, D. H. (2022). Polycystic ovary syndrome as a plausible evolutionary outcome of metabolic adaptation. Reproductive Biology and Endocrinology, 20(1). https://doi.org/10.1186/s12958-021-00878-y

Dunaif, A. (1997). Insulin resistance and the polycystic ovary Syndrome: Mechanism and implications for pathogenesis*. Endocrine Reviews, 18(6), 774–800. https://doi.org/10.1210/edrv.18.6.0318

Dyatlova, N. (2023, April 17). Metformin-Associated Lactic Acidosis (MALA). StatPearls - NCBI Bookshelf. https://www.ncbi.nlm.nih.gov/books/NBK580485/

Ehrmann, D. A., Barnes, R. B., Rosenfield, R. L., Cavaghan, M. K., & Imperial, J. (1999). Prevalence of impaired glucose tolerance and diabetes in women with polycystic ovary syndrome. Diabetes care, 22(1), 141-146.

Elkamhawy, A., Kim, N. Y., Hassan, A. H., Park, J., Paik, S., Yang, J. W., Oh, K., Lee, B. H., Lee, M. Y., Shin, K. J., Pae, A. N., Lee, K., & Roh, E. J. (2020). Thiazolidine-2,4-dione-based irreversible allosteric IKK-β kinase inhibitors: Optimization into in vivo active anti-inflammatory agents. European Journal of Medicinal Chemistry, 188, 111955. https://doi.org/10.1016/j.ejmech.2019.111955

Galić, S., Hauser, C., Kahn, B. B., Haj, F. G., Neel, B. G., Tonks, N. K., & Tiganis, T. (2005). Coordinated regulation of insulin signaling by the protein tyrosine phosphatases PTP1B and TCPTP. Molecular and Cellular Biology, 25(2), 819–829. https://doi.org/10.1128/mcb.25.2.819-829.2005

Gao, Z., Hwang, D., Bataille, F., Lefevre, M., York, D. A., Quon, M. J., & Ye, J. (2002). Serine Phosphorylation of insulin receptor substrate 1 by inhibitor ΚB kinase complex. Journal of Biological Chemistry, 277(50), 48115–48121. https://doi.org/10.1074/jbc.m209459200

Grout, B. D. M. (2019, October 15). Natural Diabetes Treatments - Berberine. Arizona Integrative Medical Center. https://www.arizonaadvancedmedicine.com/articles/2019/october/natural-diabetes-treatments-berberine/

Hardy, O. T., Czech, M. P., & Corvera, S. (2012). What causes the insulin resistance underlying obesity? Current Opinion in Endocrinology, Diabetes and Obesity, 19(2), 81–87. https://doi.org/10.1097/med.0b013e3283514e13

Huang, H., He, Y., Wan, L., Wei, W., Li, Y., Xie, R., Guo, S., Wang, Y., Jiang, J., Chen, B., Liu, J., Zhang, N., Chen, L., & He, W. (2016). Identification of polycystic ovary syndrome potential drug targets based on pathobiological similarity in the protein-protein interaction network. Oncotarget, 7(25), 37906–37919. https://doi.org/10.18632/oncotarget.9353

Khan, M. J., Ullah, A., & Basit, S. (2019). Genetic basis of polycystic ovary syndrome (PCOS): current perspectives. The application of clinical genetics, 249-260. https://doi.org/10.2147/TACG.S200341

Kumar, S., Paul, P., Yadav, P., Kaul, R., Maitra, S. S., Jha, S. K., & Chaari, A. (2022). A multi-targeted approach to identify potential flavonoids against three targets in the SARS-CoV-2 life cycle. Computers in biology and medicine, 142, 105231.

Li, A., Cong, L., Xie, F., Jin, M., & Feng, L. (2022). Berberine ameliorates insulin resistance by inhibiting IKK/NF-ΚB, JNK, and IRS-1/AKT signaling pathway in liver of gestational diabetes mellitus rats. Metabolic Syndrome and Related Disorders, 20(8), 480–488. https://doi.org/10.1089/met.2022.0017

Liu, R., Mathieu, C., Berthelet, J., Zhang, W., Dupret, J., & Lima, F. (2022). Human protein tyrosine phosphatase 1B (PTP1B): From Structure to Clinical Inhibitor Perspectives. International Journal of Molecular Sciences, 23(13), 7027. https://doi.org/10.3390/ijms23137027

Liu, F., Xia, Y., Parker, A., & Verma, I. M. (2012). IKK biology. Immunological Reviews, 246(1), 239–253. https://doi.org/10.1111/j.1600-065x.2012.01107.x

Lujan, M. E., Chizen, D. R., & Pierson, R. A. (2008). Diagnostic Criteria for Polycystic Ovary Syndrome: Pitfalls and Controversies. Journal of Obstetrics and Gynaecology Canada, 30(8), 671–679. https://doi.org/10.1016/s1701-2163(16)32915-2

Merviel, P., James, P., Bouée, S., Guillou, M. L., Rince, C., Nachtergaele, C., & Kerlan, V. (2021). Impact of myo-inositol treatment in women with polycystic ovary syndrome in assisted reproductive technologies. Reproductive Health, 18(1). https://doi.org/10.1186/s12978-021-01073-3

Moghetti, P. (2016). Insulin resistance and polycystic ovary syndrome. Current pharmaceutical design, 22(36), 5526-5534.

Moniuszko, S. (2023, June 15). What is berberine? Experts explain the supplement trending for weight loss. CBS News. https://www.cbsnews.com/news/what-is-berberine-experts-trending-weight-loss-supplement/

National Institute of Diabetes and Digestive and Kidney Diseases. (2018, June 6). Pioglitazone. LiverTox - NCBI Bookshelf. https://www.ncbi.nlm.nih.gov/books/NBK548327/#:~:text=The%20liver%20injury%20from%20pioglitazone,not%20been%20associated%20with%20pioglitazone.

National Institute of Diabetes and Digestive and Kidney Diseases. (2020, October 6). Berberine. LiverTox - NCBI Bookshelf. https://www.ncbi.nlm.nih.gov/books/NBK564659/#:~:text=Hepatotoxicity,test%20results%20in%20any%20detail.

Ndefo, U. A., Eaton, A., & Green, M. R. (2013). Polycystic ovary syndrome: a review of treatment options with a focus on pharmacological approaches. P & T : A Peer-Reviewed Journal for Formulary Management, 38(6), 336–355. https://www.ncbi.nlm.nih.gov/pmc/articles/PMC3737989/

NHS. (2022, March 24). Side effects of metformin. nhs.uk. https://www.nhs.uk/medicines/metformin/side-effects-of-metformin/#:~:text=Long%2Dterm%20side%20effects,vitamin%20B12%20supplements%20will%20help.

Panda, P. K., Rane, R., Ravichandran, R., Singh, S., & Panchal, H. (2016). Genetics of PCOS: A systematic bioinformatics approach to unveil the proteins responsible for PCOS. Genomics data, 8, 52-60.

Pandey, M. K., Sung, B., Kunnumakkara, A. B., Sethi, G., Chaturvedi, M. M., & Aggarwal, B. B. (2008). Berberine modifies cysteine 179 of IΚBΑ kinase, suppresses nuclear Factor-ΚB–Regulated antiapoptotic gene products, and potentiates apoptosis. Cancer Research, 68(13), 5370–5379. https://doi.org/10.1158/0008-5472.can-08-0511

Pattnaik, L., Naaz, S. A., Das, B., Dash, P., & Pattanaik, M. (2022). Adverse pregnancy outcome in polycystic ovarian Syndrome: a comparative study. Cureus. https://doi.org/10.7759/cureus.25790

Paulo, Thallmair, S., Paolo Conflitti, Ramírez-Palacios, C., Alessandri, R., Raniolo, S., Limongelli, V., & Marrink, S. J. (2020). Protein–ligand binding with the coarse-grained Martini model. Nature Communications, 11(1). https://doi.org/10.1038/s41467-020-17437-5

Polycystic ovary Syndrome (PCOS). (2022, February 28). Johns Hopkins Medicine. https://www.hopkinsmedicine.org/health/conditions-and-diseases/polycystic-ovary-syndrome-pcos#:~:text=Women%20with%20PCOS%20are%20more,to%20get%20pregnant%20(fertility).

Reddy, K. R., Deepika, M. L. N., Supriya, K., Latha, K. P., Rao, S. L., Rani, V. U., & Jahan, P. (2014). CYP11A1 microsatellite (tttta) n polymorphism in PCOS women from South India. Journal of assisted reproduction and genetics, 31, 857-863.

Scheen, A. (2001). Thiazolidinediones and liver toxicity. PubMed, 27(3), 305–313. https://pubmed.ncbi.nlm.nih.gov/11431595

Singh, N., & Sharma, B. (2018). Toxicological effects of berberine and sanguinarine. Frontiers in Molecular Biosciences, 5. https://doi.org/10.3389/fmolb.2018.00021

Utami, A. R., Maksum, I. P., & Deawati, Y. (2023). Berberine and its study as an antidiabetic compound. Biology, 12(7), 973. https://doi.org/10.3390/biology12070973

Verma, S. K., Yadav, Y. S., & Thareja, S. (2019). 2,4-Thiazolidinediones AS PTP 1B Inhibitors: A mini review (2012-2018). Mini-reviews in Medicinal Chemistry, 19(7), 591–598. https://doi.org/10.2174/1389557518666181026092029

Wang, K., & Li, Y. (2023). Signaling pathways and targeted therapeutic strategies for polycystic ovary syndrome. Frontiers in Endocrinology, 14. https://doi.org/10.3389/fendo.2023.1191759

Wondmkun, Y. T. (2020). Obesity, insulin resistance, and type 2 diabetes: associations and therapeutic implications. Diabetes, Metabolic Syndrome and Obesity, 3611-3616. doi: 10.2147/DMSO.S275898

Wright, J. (2017, November 10). Berberine for type 2 diabetes and more! - Tahoma Clinic. Tahoma Clinic. https://tahomaclinic.com/berberine-type-2-diabetes/

Yin, J., Huang, X., & Ye, J. (2008). Efficacy of berberine in patients with type 2 diabetes mellitus. Metabolism, 57(5), 712–717. https://doi.org/10.1016/j.metabol.2008.01.013

Zhang, S., Zhou, J., Gober, H., Leung, W. T., & Wang, L. (2021). Effect and mechanism of berberine against polycystic ovary syndrome. Biomedicine & Pharmacotherapy, 138, 111468. https://doi.org/10.1016/j.biopha.2021.111468

Zhao, H., Zhang, J., Cheng, X., Nie, X., & He, B. (2023). Insulin resistance in polycystic ovary syndrome across various tissues: an updated review of pathogenesis, evaluation, and treatment. Journal of Ovarian Research, 16(1). https://doi.org/10.1186/s13048-022-01091-0

Zhou, J., Huang, X., Xue, B., Wei, Y., & Fang, H. (2021). Bioinformatics analysis of the molecular mechanism of obesity in polycystic ovary syndrome. Aging, 13(9), 12631–12640. https://doi.org/10.18632/aging.202938